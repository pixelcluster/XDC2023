\documentclass[aspectratio=169,t]{beamer}

\usetheme{XDC}
\usepackage[utf8]{inputenc}
\usepackage{listings}

\author{Friedrich Vock}

\title{Improving the World's Slowest Raytracer}

\date{October 17, 2023}

\setBackground{graphics/frogrender.png}

\lstset{
  basicstyle=\Tiny\ttfamily,
  columns=fullflexible,
}

\chapterIntroConfig

\begin{document}

\begin{slide}
\titlepage
\end{slide}

\defaultConfig

\begin{slide}{Contents}
\tableofcontents
\end{slide}

\section{Raytracing: A Quick Recap}

\chapterIntroConfig
\begin{slide}{Raytracing: A Quick Recap}
\end{slide}
\defaultConfig

\subsection{Acceleration Structures}
\begin{slide}{Acceleration Structures}
    \begin{itemize}
     \item Opaque from the app's POV
     \item Contains scene geometry in format understood by ray accelerator HW
     \item Built by the driver, app provides ``primitive soup''
     \item Three primitive types: Triangles, AABBs and instances of other BVHs
     \begin{itemize}
      \item With AABB primitives, a shader can execute custom code to determine if a ray hits the surface
      \item Vulkan defines two levels of acceleration structures: Top-Level and Bottom-Level
      \begin{itemize}
        \item TLAS can only have BLAS instances as primitives
        \item BLAS can only have triangles or AABBs as primitives
      \end{itemize}
     \end{itemize}
    \end{itemize}
\end{slide}
\subsection{Raytracing Pipelines}

\begin{slide}{Raytracing Pipelines}
    \begin{itemize}
     \item Special type of pipeline for performing raytracing
     \item New shader stages: Ray Generation, Any Hit, Intersection, Closest Hit, Miss, Callable
    \end{itemize}
    \pause
    \includegraphics[width=0.95\linewidth]{graphics/RTStages1.png}
\end{slide}

\begin{slide}{Raytracing Pipelines}
    \begin{itemize}
     \item Special type of pipeline for performing raytracing
     \item New shader stages: Ray Generation, Any Hit, Intersection, Closest Hit, Miss, Callable
    \end{itemize}
    \includegraphics[width=0.95\linewidth]{graphics/RTStages2.png}
\end{slide}

\begin{slide}{Raytracing Pipelines}
    \begin{itemize}
     \item Special type of pipeline for performing raytracing
     \item New shader stages: Ray Generation, Any Hit, Intersection, Closest Hit, Miss, Callable
    \end{itemize}
    \includegraphics[width=0.95\linewidth]{graphics/RTStages3.png}
\end{slide}

\begin{slide}{Raytracing Pipelines}
    \begin{itemize}
     \item Special type of pipeline for performing raytracing
     \item New shader stages: Ray Generation, Any Hit, Intersection, Closest Hit, Miss, Callable
    \end{itemize}
    \includegraphics[width=0.95\linewidth]{graphics/RTStages4.png}
\end{slide}

\begin{slide}{Raytracing Pipelines}
    \begin{itemize}
     \item Special type of pipeline for performing raytracing
     \item New shader stages: Ray Generation, Any Hit, Intersection, Closest Hit, Miss, Callable
    \end{itemize}
    \includegraphics[width=0.95\linewidth]{graphics/RTStages5.png}
\end{slide}

\begin{slide}{Raytracing Pipelines}
    \begin{itemize}
     \item Special type of pipeline for performing raytracing
     \item New shader stages: Ray Generation, Any Hit, Intersection, Closest Hit, Miss, Callable
    \end{itemize}
    \includegraphics[width=0.95\linewidth]{graphics/RTStages6.png}
\end{slide}

\begin{slide}{Raytracing Pipelines}
    \begin{itemize}
     \item Special type of pipeline for performing raytracing
     \item New shader stages: Ray Generation, Any Hit, Intersection, Closest Hit, Miss, Callable
    \end{itemize}
    \includegraphics[width=0.95\linewidth]{graphics/RTStages7.png}
\end{slide}

\begin{slide}{Raytracing Pipelines}
    \begin{itemize}
     \item Special type of pipeline for performing raytracing
     \item New shader stages: Ray Generation, Any Hit, Intersection, Closest Hit, Miss, Callable
    \end{itemize}
    \includegraphics[width=0.95\linewidth]{graphics/RTStages8.png}
\end{slide}

\begin{slide}{Raytracing Pipelines}
    \begin{itemize}
     \item Special type of pipeline for performing raytracing
     \item New shader stages: Ray Generation, Any Hit, Intersection, Closest Hit, Miss, Callable
    \end{itemize}
    \includegraphics[width=1.0651\linewidth]{./graphics/RTStages9.png}
    \pause
    \begin{itemize}
    \item Callable shaders can be called from RGen/CHit/Miss(/Callable)
    \end{itemize}
\end{slide}

\begin{slide}{Other RT Pipeline concepts}
    \begin{itemize}
      \item Shader Binding Tables (SBTs)
      \begin{itemize}
       \item More than one shader for each stage!
       \item App tells the driver which shader to call with a Shader Binding Table
       \item Jump Table on the GPU
      \end{itemize}
      \item Pipeline Libraries
      \begin{itemize}
       \item Incomplete pipelines that can be linked with other pipelines
       \item Useful for e.g. moving commonly-used shaders into common library
      \begin{itemize}
       \item Pipelines using these shaders only need to link to the previously-created library
      \end{itemize}
      \end{itemize}

    \end{itemize}
\end{slide}

\section{Implementing Acceleration Structures}

\chapterIntroConfig
\begin{slide}{Implementing Acceleration Structures}
\end{slide}
\defaultConfig

\begin{slide}{Acceleration Structure Building}
 \begin{itemize}
  \item RADV implements acceleration structures as BVHs
  \begin{itemize}
    \item Tree of AABBs and triangles
    \item Internal nodes are AABBs, each leaf node represents a primitive
    \item A node's AABB is the AABB of its child nodes
  \end{itemize}
  \item Hardware format is BVH4
  \begin{itemize}
    \item Each internal node can have up to 4 child nodes
  \end{itemize}
 \end{itemize}

 \begin{minipage}[t]{0.45\textwidth}
 \vspace*{2pt}
 Node ID packing:
 \begin{itemize}
  \item Node offset in BVH / 8
  \item 3 LSBs used for node type
 \end{itemize}

% now that's a way to write code
\texttt{struct radv\_bvh\_box32\_node \{\\
\hspace*{8pt} uint32\_t children[4];\\
\hspace*{8pt} radv\_aabb coords[4];\\
\hspace*{8pt} uint32\_t reserved[4];\\
\};}
 \end{minipage}
 \begin{minipage}[t]{0.5\textwidth}
\texttt{struct radv\_bvh\_triangle\_node \{}\\
\texttt{\hspace*{8pt} float coords[3][3];}\\
\texttt{\hspace*{8pt} uint32\_t reserved[3];}\\
\texttt{\hspace*{8pt} uint32\_t triangle\_id;}\\
\texttt{\hspace*{8pt} /* flags in upper 4 bits */}\\
\texttt{\hspace*{8pt} uint32\_t geometry\_id\_and\_flags;}\\
\texttt{\hspace*{8pt} uint32\_t reserved2;}\\
\texttt{\hspace*{8pt} uint32\_t id;}\\
\texttt{\};}
 \end{minipage}
\end{slide}


\begin{slide}{Acceleration Structure Building}
 \begin{itemize}
  \item Building acceleration structures is a multi-step process
  \begin{enumerate}
   \item Convert input primitives to leaf nodes
   \item Sort nodes
   \item Construct binary BVH
   \item Convert binary BVH to BVH4
  \end{enumerate}
  \item Need barriers in between each step!
  \begin{itemize}
   \item Bad for building many tiny acceleration structures
   \item Vulkan allows for building multiple acceleration structures at a time
  \end{itemize}
 \end{itemize}
\end{slide}

\begin{slide}{Acceleration Structure Building}
 \begin{itemize}
  \item Building acceleration structures is a multi-step process
  \begin{enumerate}
   \item Convert input primitives to leaf nodes
   \item Sort nodes
   \item Construct binary BVH
   \item Convert binary BVH to BVH4
  \end{enumerate}
  \item Need barriers in between each step!
  \begin{itemize}
   \item Bad for building many tiny acceleration structures
   \item Vulkan allows for building multiple acceleration structures at a time
  \end{itemize}
  \pause
  \item This is what happens when you don't batch:
 \end{itemize}
 \includegraphics[width=\textwidth]{graphics/mostperformantasbuild.png}
\end{slide}

\chapterIntroConfig
\begin{slide}{BVH Building Algorithms}
\end{slide}
\defaultConfig

\subsection{LBVH}

\begin{slide}{Linear BVH}
  \begin{itemize}
   \item Algorithm from 2014
   \item Builds a radix tree from generated morton codes
   \item Tree can be built up independently for each node
   \begin{itemize}
    \item Great for parallelization!
   \end{itemize}
   \item Used for quick builds
  \end{itemize}
\end{slide}

\subsection{PLOC}

\begin{slide}{Parallel Locally-Ordered Clustering}
  \begin{itemize}
   \item Algorithm from 2017
   \item Better quality than LBVH, though slower
   \item Original reference implementation was written in CUDA
   \item Heavily relies on device-wide synchronization
   \begin{itemize}
    \item No global synchronization primitives available
    \item Have to make our own!
   \end{itemize}
  \end{itemize}
\end{slide}

\begin{slide}{The Poor Man's Global Synchronization}
  \begin{itemize}
   \item All build shaders are written in Vulkan GLSL
   \item Compiler and target hardware is known
   \item Can assume more guarantees in certain aspects
   \begin{itemize}
    \item Forward Progress
    \item Launch ordering
   \end{itemize}
   \item Work Stealing Queue
   \begin{itemize}
    \item Each task receives a unique ID
    \item Runtime-defined number of tasks to launch before synchronizing
   \end{itemize}
   Launching new tasks:\\
   \pause
   \texttt{taskID = atomicAdd(taskStartedCounter, 1);}\\
  \end{itemize}
\end{slide}

\begin{slide}{The Poor Man's Global Synchronization}
   Making sure previous work has finished:\\
  \begin{itemize}
   \item Maintain a ``tasks finished'' counter and increment it when finished
   \item If the new work ID is larger than the maximum task ID to launch before synchronizing, busy-wait until the maximum task ID is updated
   \item The last task to finish updates the maximum task ID
  \end{itemize}

  \texttt{finishID = atomicAdd(taskFinishCounter, 1);\\
  if (finishID == maxTaskID) \{\\
  \hspace*{8pt}atomicAdd(maxTaskID, nextNumberOfTasks);\\
  \}
  else \{\\
  \hspace*{8pt}while (taskID > maxTaskID) \{\}\\
  \}
  }\\
\end{slide}

\begin{frame}[fragile]{Totally sane BVH building code}
\small
  \begin{lstlisting}[columns=fullflexible]
controlBarrier(gl_ScopeWorkgroup, gl_ScopeDevice, gl_StorageSemanticsBuffer,
               gl_SemanticsAcquireRelease | gl_SemanticsMakeAvailable | gl_SemanticsMakeVisible);
if (gl_LocalInvocationIndex == 0) {
   if (did_work)
      atomicAdd(DEREF(header).sync_data.task_done_counter, 1);
   global_task_index = atomicAdd(DEREF(header).sync_data.task_started_counter, 1);

   do {
      /* Perform a memory barrier to refresh the current phase's end counter, in case
       * another workgroup changed it. */
      memoryBarrier(gl_ScopeDevice, gl_StorageSemanticsBuffer,
                    gl_SemanticsAcquireRelease | gl_SemanticsMakeAvailable | gl_SemanticsMakeVisible);

      /* The first invocation of the first workgroup in a new phase is responsible to initiate the
       * switch to a new phase. It is only possible to switch to a new phase if all tasks of the
       * previous phase have been completed. Switching to a new phase and incrementing the phase
       * end counter in turn notifies all invocations for that phase that it is safe to execute.
       */
      if (global_task_index == DEREF(header).sync_data.current_phase_end_counter &&
          DEREF(header).sync_data.task_done_counter == DEREF(header).sync_data.current_phase_end_counter) {
         if (DEREF(header).sync_data.next_phase_exit_flag != 0) {
            DEREF(header).sync_data.phase_index = TASK_INDEX_INVALID;
            memoryBarrier(gl_ScopeDevice, gl_StorageSemanticsBuffer,
                          gl_SemanticsAcquireRelease | gl_SemanticsMakeAvailable | gl_SemanticsMakeVisible);
         } else {
            atomicAdd(DEREF(header).sync_data.phase_index, 1);
            DEREF(header).sync_data.current_phase_start_counter = DEREF(header).sync_data.current_phase_end_counter;
            /* Ensure the changes to the phase index and start/end counter are visible for other
             * workgroup waiting in the loop. */
            memoryBarrier(gl_ScopeDevice, gl_StorageSemanticsBuffer,
                          gl_SemanticsAcquireRelease | gl_SemanticsMakeAvailable | gl_SemanticsMakeVisible);
            atomicAdd(DEREF(header).sync_data.current_phase_end_counter,
                      DIV_ROUND_UP(task_count(header), gl_WorkGroupSize.x));
         }
         break;
      }

      /* If other invocations have finished all nodes, break out; there is no work to do */
      if (DEREF(header).sync_data.phase_index == TASK_INDEX_INVALID) {
         break;
      }
   } while (global_task_index >= DEREF(header).sync_data.current_phase_end_counter);

   shared_phase_index = DEREF(header).sync_data.phase_index;
 }

barrier();
if (DEREF(header).sync_data.phase_index == TASK_INDEX_INVALID)
   return TASK_INDEX_INVALID;

num_tasks_to_skip = shared_phase_index - phase_index;

uint32_t local_task_index = global_task_index - DEREF(header).sync_data.current_phase_start_counter;
return local_task_index * gl_WorkGroupSize.x + gl_LocalInvocationID.x;
  \end{lstlisting}
  \normalsize
\end{frame}

\section{Implementing Raytracing Pipelines}

\chapterIntroConfig
\begin{slide}{Implementing Raytracing Pipelines}
\end{slide}
\defaultConfig

\begin{slide}{Hardware Acceleration Features on AMD}
 \begin{itemize}
  \item Instruction for hardware-accelerated ray-BVH intersection: \texttt{image\_bvh\_intersect\_ray}
  \begin{itemize}
   \item Calculates intersection results for a single BVH node
   \item Returns child nodes sorted by intersection distance for internal AABB nodes
   \item Returns intersection results (incl. barycentrics for triangles) for leaf nodes
   \item New in RDNA3: Ray flag handling and more box sorting modes
  \end{itemize}
 \end{itemize}
 \begin{itemize}
  \item New in RDNA3: Instruction for LDS traversal stack: \texttt{ds\_bvh\_stack\_rtn}
  \begin{itemize}
   \item Optimization for stack handling
   \item Not used in RADV (yet)
  \end{itemize}
 \end{itemize}
\end{slide}

\begin{slide}{Hardware Acceleration Features on AMD}
 \begin{itemize}
  \item Instruction for hardware-accelerated ray-BVH intersection: \texttt{image\_bvh\_intersect\_ray}
  \begin{itemize}
   \item Calculates intersection results for a single BVH node
   \item Returns child nodes sorted by intersection distance for internal AABB nodes
   \item Returns intersection results (incl. barycentrics for triangles) for leaf nodes
   \item New in RDNA3: Ray flag handling and more box sorting modes
  \end{itemize}
 \end{itemize}
 \begin{itemize}
  \item New in RDNA3: Instruction for LDS traversal stack: \texttt{ds\_bvh\_stack\_rtn}
  \begin{itemize}
   \item Optimization for stack handling
   \item Not used in RADV (yet)
  \end{itemize}
 \item All the rest: Implemented in software
 \end{itemize}

 \vspace*{-40pt}
 \hspace*{240pt}
 \includegraphics{graphics/booksmeme.png}
\end{slide}

\begin{slide}{Layout of a Raytracing pipeline - XDC 2022}
\begin{itemize}
 \item Inline all raytracing shaders into a single compute shader
 \item Shaders are referred to by IDs
 \item Wrapped in a loop that continues executing shaders unless exit is requested
\end{itemize}

  \texttt{void main() \{ \\
  \hspace*{8pt}uint32\_t shader\_id = raygen\_sbt[0]; \\
  \hspace*{8pt}while (shader\_id != 0) \{ \\
  \hspace*{16pt}switch (shader\_id) \{ \\
  \hspace*{24pt}// paste all shader code in here \\
  \hspace*{16pt}\} \\
  \hspace*{8pt}\} \\
  \}} \\
\end{slide}

\begin{slide}{Layout of a Raytracing pipeline - XDC 2022}
 Problem: Raytracing pipelines can get \textbf{big} (2000+ shaders)
 \begin{itemize}
  \item Size of shaders caused stack overflows during passes that operated recursively on blocks
  \item Does not work well with pipeline libraries - have to copy shader code around and defer actual compilation
 \end{itemize}
\end{slide}

\begin{slide}{Layout of a Raytracing pipeline - XDC 2023}
 ``Monolithic'' mode:
 \begin{itemize}
  \item All shaders are fully inlined
  \item No shader call overhead, potential for constant propagation
  \item No recursion, no support for pipeline libraries
 \end{itemize}
  % epicer LaTeX programming,
  \texttt{void traceRay() // always inlined \{ \\
  \hspace*{8pt}// paste traversal shader \\
  \hspace*{8pt}if (hit) \\
  \hspace*{16pt}// paste closest-hit shader; \\
  \hspace*{8pt}else \\
  \hspace*{16pt}// paste miss shader \\
  \}} \\
  \texttt{void main() \{ \\
  \hspace*{8pt}// paste raygen shader \\
  \}} \\
\end{slide}

\begin{slide}{Layout of a Raytracing pipeline - XDC 2023}
 ``Separate Compilation'' mode:
 \begin{itemize}
  \item Shader calls with indirect jump
  \item Allows shaders from pipeline libraries to be compiled ahead of time
 \end{itemize}

 \includegraphics[width=1.0651\linewidth]{graphics/RTStages10.png}

\end{slide}

\begin{slide}{Layout of a Traversal Shader}
  \vspace*{-2pt} % sad hack, otherwise it doesn't fit :(
                 % nobody will notice
  % epic LaTeX programming
  \texttt{void main() \{ \\
  \hspace*{8pt}while (!traversal\_complete) \{ \\
  \hspace*{16pt}uint32\_t node\_pointer = get\_next\_node(); \\
  \hspace*{16pt}intersection\_result result = image\_bvh\_intersect\_ray(node\_pointer); \\
  \hspace*{16pt}if (result.hit) \{ \\
  \hspace*{24pt}// paste any-hit/intersection shaders here \\
  \hspace*{24pt}if (!hit\_accepted) \\
  \hspace*{32pt}continue; \\
  \hspace*{16pt}\} \\
  \hspace*{8pt}\} \\
  \hspace*{8pt}if (hit) \\
  \hspace*{16pt}call\_closest\_hit(); \\
  \hspace*{8pt}else \\
  \hspace*{16pt}call\_miss(); \\
  \}} \\
\end{slide}

\begin{slide}{Why not separately-compile the traversal shader?}
 \begin{itemize}
  \item Ray Traversal has lots of live state
  \item Any-Hit/Intersection shaders are relatively small yet called often
  \item High shader call overhead, negatively impacts performance
 \end{itemize}
\end{slide}

\begin{slide}{Implementing Indirect Jumps}
 \begin{itemize}
  \item Jump execution to arbitrary addresses stored in SBT
  \item Each invocation in a wavefront may have a different address
  \item Only one program counter per wavefront
  \item Naive solution: Choose program counter of first valid invocation
  \item Guard shader invocations to prevent invocations from executing wrong shaders
  \item Problem: Reconvergent shader calls become inefficient!
 \end{itemize}
 \pause
 \includegraphics[width=\textwidth]{graphics/RTStages2-1.png}
\end{slide}

\begin{slide}{Implementing Indirect Jumps}
 \begin{itemize}
  \item Jump execution to arbitrary addresses stored in SBT
  \item Each invocation in a wavefront may have a different address
  \item Only one program counter per wavefront
  \item Naive solution: Choose program counter of first valid invocation
  \item Guard shader invocations to prevent invocations from executing wrong shaders
  \item Problem: Reconvergent shader calls become inefficient!
 \end{itemize}
 \includegraphics[width=\textwidth]{graphics/RTStages2-2.png}
\end{slide}

\begin{slide}{Implementing Indirect Jumps}
 \begin{itemize}
  \item Jump execution to arbitrary addresses stored in SBT
  \item Each invocation in a wavefront may have a different address
  \item Only one program counter per wavefront
  \item Naive solution: Choose program counter of first valid invocation
  \item Guard shader invocations to prevent invocations from executing wrong shaders
  \item Problem: Reconvergent shader calls become inefficient!
 \end{itemize}
 \includegraphics[width=\textwidth]{graphics/RTStages2-3.png}
\end{slide}

\begin{slide}{Implementing Indirect Jumps}
 \begin{itemize}
  \item Jump execution to arbitrary addresses stored in SBT
  \item Each invocation in a wavefront may have a different address
  \item Only one program counter per wavefront
  \item Naive solution: Choose program counter of first valid invocation
  \item Guard shader invocations to prevent invocations from executing wrong shaders
  \item Problem: Reconvergent shader calls become inefficient!
 \end{itemize}
 \includegraphics[width=\textwidth]{graphics/RTStages2-4.png}
\end{slide}

\begin{slide}{Implementing Indirect Jumps}
 \begin{itemize}
  \item Jump execution to arbitrary addresses stored in SBT
  \item Each invocation in a wavefront may have a different address
  \item Only one program counter per wavefront
  \item Naive solution: Choose program counter of first valid invocation
  \item Guard shader invocations to prevent invocations from executing wrong shaders
  \item Problem: Reconvergent shader calls become inefficient!
 \end{itemize}
 \includegraphics[width=\textwidth]{graphics/RTStages2-5.png}
\end{slide}

\begin{slide}{Implementing Indirect Jumps}
 \begin{itemize}
  \item Jump execution to arbitrary addresses stored in SBT
  \item Each invocation in a wavefront may have a different address
  \item Only one program counter per wavefront
  \item Naive solution: Choose program counter of first valid invocation
  \item Guard shader invocations to prevent invocations from executing wrong shaders
  \item Problem: Reconvergent shader calls become inefficient!
 \end{itemize}
 \includegraphics[width=\textwidth]{graphics/RTStages2-6.png}
\end{slide}

\begin{slide}{Implementing Indirect Jumps}
 \begin{itemize}
  \item Solution: Select shader based on type
  \item Certain shader types will always be preferred over others
  \item Prefer Hit/Miss over Traversal shaders
  \item Execute Raygen only when no other shaders are available
  \item Shader type is packed into the 2 LSBs of the shader's address
 \end{itemize}
 \includegraphics[width=\textwidth]{graphics/RTStages2-1.png}
\end{slide}

\begin{slide}{Implementing Indirect Jumps}
 \begin{itemize}
  \item Solution: Select shader based on type
  \item Certain shader types will always be preferred over others
  \item Prefer Hit/Miss over Traversal shaders
  \item Execute Raygen only when no other shaders are available
  \item Shader type is packed into the 2 LSBs of the shader's address
 \end{itemize}
 \includegraphics[width=\textwidth]{graphics/RTStages2-2.png}
\end{slide}

\begin{slide}{Implementing Indirect Jumps}
 \begin{itemize}
  \item Solution: Select shader based on type
  \item Certain shader types will always be preferred over others
  \item Prefer Hit/Miss over Traversal shaders
  \item Execute Raygen only when no other shaders are available
  \item Shader type is packed into the 2 LSBs of the shader's address
 \end{itemize}
 \includegraphics[width=\textwidth]{graphics/RTStages2-3.png}
\end{slide}

\begin{slide}{Implementing Indirect Jumps}
 \begin{itemize}
  \item Solution: Select shader based on type
  \item Certain shader types will always be preferred over others
  \item Prefer Hit/Miss over Traversal shaders
  \item Execute Raygen only when no other shaders are available
  \item Shader type is packed into the 2 LSBs of the shader's address
 \end{itemize}
 \includegraphics[width=\textwidth]{graphics/RTStages2-7.png}
\end{slide}

\begin{slide}{Implementing Indirect Jumps}
 \begin{itemize}
  \item Solution: Select shader based on type
  \item Certain shader types will always be preferred over others
  \item Prefer Hit/Miss over Traversal shaders
  \item Execute Raygen only when no other shaders are available
  \item Shader type is packed into the 2 LSBs of the shader's address
 \end{itemize}
 \includegraphics[width=\textwidth]{graphics/RTStages2-8.png}
\end{slide}

\section{Status And What's To Come}

\chapterIntroConfig
\begin{slide}{Status And What's To Come}
\end{slide}
\defaultConfig

\subsection{Where We Are}
\begin{slide}{Compatibility}
 XDC 2022: \\
 \begin{itemize}
  \item Quake II RTX
  \item Control
  \item Deathloop
  \item Resident Evil: Village
  \item Metro Exodus Enhanced Edition
 \end{itemize}

 \pause

 \vspace{12pt}

 XDC 2023: \\
 \begin{itemize}
  \item General expectation is that new stuff works
  \item Some known issues (Witcher 3, Cyberpunk 2077 can hang currently)
  \item Something else broken?
  \begin{itemize}
   \item https://gitlab.freedesktop.org/mesa/mesa/-/issues/new
  \end{itemize}
 \end{itemize}
\end{slide}

\begin{slide}{Performance}
\begin{center}
  \includegraphics[width=0.95\textwidth]{graphics/perfcomparison-rtreflections.png}
\end{center}
\end{slide}

\subsection{Where We're Going}
\begin{slide}{Future Work}
 \begin{itemize}
  \item Make remaining games work
  \item Make remaining games work \textbf{fast}
  \item Explore more sophisticated shader call techniques
  \begin{itemize}
   \item Shader Execution Reordering?
  \end{itemize}
  \item Make the monolithic path work with pipeline libraries
  \item Take advantage of RDNA3 features
  \item Build higher quality BVHs
  \item AS Updates
  \item Tons of microoptimizations
 \end{itemize}

\end{slide}

\chapterIntroConfig
\begin{slide}{ }
 \begin{center}
 % wtf??? normal text should be white
 \textcolor{white}{
 \huge
 \textbf{Thank You!}\\ \vspace{6pt}
 \normalsize
 Questions?}
 \end{center}
\end{slide}
\defaultConfig


\end{document}

